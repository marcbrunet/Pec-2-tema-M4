\documentclass[preprint,11pt]{elsarticle}


\usepackage{fullpage} % Package to use full page
\usepackage{parskip} % Package to tweak paragraph skipping
\usepackage{tikz} % Package for drawing
\usepackage{amsmath}
\usepackage{hyperref}
\usepackage{xcolor}
\usepackage{listings}
\usepackage{multicol}
%New colors defined below
\definecolor{codegreen}{rgb}{0,0.6,0}
\definecolor{codegray}{rgb}{0.5,0.5,0.5}
\definecolor{codepurple}{rgb}{0.58,0,0.82}
\definecolor{backcolour}{rgb}{0.95,0.95,0.92}
\lstdefinestyle{mystyle}{
  backgroundcolor=\color{backcolour},   commentstyle=\color{codegreen},
  keywordstyle=\color{magenta},
  numberstyle=\tiny\color{codegray},
  stringstyle=\color{codepurple},
  basicstyle=\footnotesize,
  breakatwhitespace=false,         
  breaklines=true,                 
  captionpos=b,                    
  keepspaces=true,                 
  numbers=left,                    
  numbersep=5pt,                  
  showspaces=false,                
  showstringspaces=false,
  showtabs=false,                  
  tabsize=2
}
%"mystyle" code listing set
\lstset{style=mystyle}
\journal{PEC2}
\begin{document}
\begin{frontmatter}

    \title{Pec 2: tema M4}
    \author{Marc brunet presas}
    \address{Manresa, Barcelona,}
    \begin{abstract}
    Esta PEC está orientada a profundizar en aspectos de las tareas vinculadas a la administración local de una máquina. Es el primer contacto del administrador en relación a las tareas de obtención de información, configuración y mantenimiento de las máquinas para conocer que ocurre en la máquina y proveer los servicios necesarios para sus usuarios.

    \end{abstract}
\end{frontmatter}

\section{El departamento técnico ya ha comprado un servidor, de modelo PC de tipo torre. Dispone de 4 bahías de disco S-ATA, pero de momento sin ninguna unidad de disco instalada. Teniendo en cuenta los precios actuales de discos duros internos, analizar qué configuración de discos se debería comprar para disponer de:}

\subsection{}

\end{document}
